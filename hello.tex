%!TEX program = xelatex

% 导言区
\documentclass{article} 

% 导入中文宏
\usepackage{ctex}
\newcommand\degree{^\circ}
\usepackage{listings}
\usepackage{color}
\usepackage{geometry}
\geometry{a4paper,scale=0.8}


\definecolor{dkgreen}{rgb}{0,0.6,0}
\definecolor{gray}{rgb}{0.5,0.5,0.5}
\definecolor{mauve}{rgb}{0.58,0,0.82}

\lstset{frame=tb,
  language=C,
  aboveskip=3mm,
  belowskip=3mm,
  showstringspaces=false,
  columns=flexible,
  basicstyle={\small\ttfamily},
  numbers=none,
  numberstyle=\tiny\color{gray},
  keywordstyle=\color{blue},
  commentstyle=\color{dkgreen},
  stringstyle=\color{mauve},
  breaklines=true,
  breakatwhitespace=true
  tabsize=3
}


\author{\kaishu }
\date{\today}

\begin{document}

\section{第一章}


    \[y=sin(x) + cos(x)\]
    
\section{循环}
    \subsection*{while 循环输出平方数}
        \begin{enumerate}
            \item 打印出小于某数的平方数
                \begin{lstlisting}
#include<stdio.h>

int main(void){
    int i, n;
    printf("This program prints a table of squares.\n");
    printf("Enter number of entries in table: \n");
    scanf("%d", &n);

    i = 1;
    while (i<=n){
        printf("%10d%10d\n", i, i * i);
        i++;
    }
    return 0;
}
                \end{lstlisting}

            \item 数列求和: sum.c
                \begin{lstlisting}
/* 数列求和*/

#include<stdio.h>

int main(void){
    int n, sum = 0;
    printf("This program sums a series of integers.\n");
    printf("Enter a integers (0 to terminate): ");

    scanf("%d", &n);
    while(n != 0){
        sum += n;
        scanf("%d", &n);
    }
    printf("The sum is: "%d\n", sum);

    return 0;
}
                \end{lstlisting}

            \subsection*{do语句}
            先执行一次函数体,再进行条件判断
                \item do语句
                \begin{lstlisting}
/* 计算数字的位数 */
#include<stdio.h>
int main(void){
    int digits = 0, n;
    printf("Enther a nonnegative integer: ");
    scanf("%d", &n);

    do {
        n /= 10;
        digits++;
    } while(n > 0);

    printf("The number has %d digit(s).\n", digits);

    return 0;
}
                \end{lstlisting}

            \subsection*{for语句}
            \item for循环计算平方数
            \begin{lstlisting}
include<stdio.h>
int main(void)
int i;
printf("This program prints a table of squares. \n")
printf("Enter number of entries in table: ");
scanf("d", &n);

for (i =1: i <=n: i++)
    printf("%10d%10d\n", i, i * i);

return 0
            \end{lstlisting}

            \item break语句,跳出当前循环
            
            \begin{lstlisting}
/*  读入数字,j计算之和,遇到0则截止 */
#include<stdio.h>

int main(void){
    int n = 0, sum = 0;
    int i;
    printf("Enter a series of integer.\n");
    while (n < 5) {
        scanf("%d", &i);
        if (i == 0)
            break;
        sum += i;
        n++;
    }
    printf("The sum of integers is: %d",  sum);   
}               
            
            \end{lstlisting}
            
            \item continue语句,跳出当次循环,继续执行剩余循环
            
            \begin{lstlisting}
/*  读入数字,计算之和,遇到0,跳过当次循环,继续执行下次循环 */
#include<stdio.h>

int main(void){
    int n = 0, sum = 0;
    int i;
    printf("Enter a series of integer.\n");
    while (n < 5) {
        scanf("%d", &i);
        if (i == 0)
            continue;
        sum += i;
        n++;
    }
    printf("The sum of integers is: %d",  sum);   
}
            \end{lstlisting}
            
    
\end{enumerate}

\section*{数组}
\subsection*{一维数组}
\begin{enumerate}

\item 数组初始化
\item 反向输出数列
    
\begin{lstlisting}
// 反向输出数组
#include<stdio.h>
#define N 5

int main(void){
    int a[N], i;
    printf("Enter %d numbers: ", N);
    for (i = 0; i < N; i++)
        scanf("%d",  &a[i]);
    printf("In reverse order:");
    
    for (i = N - 1; i >= 0; i--)
        printf(" %d", a[i]);
    printf("\n");

    return 0;
}
\end{lstlisting}

\item 数组初始化:如果初始化比数组短,后续元素默认为0

\begin{lstlisting}
int a[10] = {1,2,3,4};
// int a[10] = {1,2,3,4,5,0,0,0,0,0} 
int a[10]={0};
// int a[10] = {0,0,0,0,0,0,0,0,0,0};
// 给定了所有元素,可省去个数
int a[] = {1,2,3,4,5,6,7,8,9,10};
// 指定位置元素为0,C99
int a[15] = {[2] = 29, [8] = 7, [14] = 100};
// 数组长度为24
int a[] = {[2] = 2, [23] = 11, [6] = 56};
\end{lstlisting}

\item 检查一个数是否有重复数字

\begin{lstlisting}
// 检查数有没有重复数字
#include<stdbool.h> // only for c99
#include<stdio.h>
int main(void){
    bool digit_seen[10] = {false};
    int digit;
    long n;

    printf("Enter a number: ");
    scanf("%ld", &n);

    while (n > 0) {
        digit = n % 10;
        if (digit_seen[digit])
            break;
        digit_seen[digit] = true;
        n /=  10;
    }

    if (n > 0)
        printf("Repeated digit\n");
    else
        printf("No repeated  digit\n");
    return  0;
}
\end{lstlisting}

\end{enumerate}




\end{document}
